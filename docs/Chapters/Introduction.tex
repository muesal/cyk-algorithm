\section{Introduction}

The question whether a given string is in the language of a given context free grammar (CFG) is known as the membership problem.
A CFG contains a set of variables and productions.
Applying the productions on the variables derives a finite set of strings, which form the language of the grammar.
Na\"{i}ve approaches to solve the membership problem have exponential running time.
In this report, we evaluate the Cocke-Younger-Kasami algorithm, which solves the problem in $O(n^3)$ for grammars that are in Chomsky normal form.
We look at a bottom-up approach and a top-down approach which uses memoization.
Further, we try to specialize the algorithm to parse strings for linear grammars.

The evaluation shows that the bottom-up and top-down algorithms behave very differently.
The running times of bottom-up are steadier than those of top-down, i.e., the running time varies less for parsing different strings of the same length for the same grammar.
We further see that the bottom-up approach behaves better on average.
Nevertheless, the top-down approach can yield a lot faster running times, depending on the order of the productions of the grammar and the input string.

However, adapting the bottom-up algorithm to parse linear grammars yields even better results.
The running time for the specialized algorithm lies in $O(n^2)$.

The CYK algorithm can further be used to not only solve the membership problem, but to also compute the minimal number of errors in the input string.
This is the minimal number of symbols in the input string that must be replaced or deleted for the string to be in the language of the grammar.
This algorithm is in $O(n^3)$ too, but in practice it is slow compared to the non-generalized algorithm.
In a second step, the error count can be used to correct the string. The corresponding algorithm is in $O(n)$.

In~\cref{sec:backgrund} we define context-free grammars and introduce the CYK algorithm, as well as the three different approaches used for the evaluation.
In~\cref{sec:specialization} we show how we specialize the algorithm for linear grammars and how the CYK-algorithm can be used to detect and correct errors.
Then, we evaluate all algorithms in~\cref{sec:Evaluation}.


