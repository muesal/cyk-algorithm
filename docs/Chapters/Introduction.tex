\section*{Introduction}

The membership problem asks, whether a given string is in the language of a given context free grammar (CFG).
A CFG holds a set of variables and productions.
Applying the productions on the variables derives a finite set of strings, which form the language of the grammar.
Na\"{i}ve approaches to solve the membership problem have exponential running time.
In this report, we evaluate the Cocke-Younger-Kasami algorithm, which solves the problem in $O(n^3)$ for grammars that are in Chomsky normal form.
We look at a bottom-up approach, as well as a top-down approach which uses memoization.
Further, we try to specialize the algorithm to parse strings for linear grammars.

The evaluation shows, that bottom-up and top-down behave very differently.
The running times of bottom-up are steadier than those of top-down, i.e., the running time varies less for parsing different strings of the same length for the same grammar.
We further see, that bottom-up behaves better on average.
Nevertheless, top-down can yield a lot faster running times, depending on the order of the productions of the grammar and the input string.

However, adapting the bottom-up algorithm to parse linear grammars yields even better results.
The running time for the specialized algorithm lies in $O(n^2)$.

In section~\ref{sec:backgrund} we define Context free grammars and introduce the CYK algorithm, as well as the three different approaches used for the evaluation.
In section~\ref{sec:specialization} we show how we generalize the algorithm for linear grammars, before we evaluate all algorithms in section~\ref{sec:Evaluation}.


