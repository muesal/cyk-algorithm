%% ----------------------------------------------------------------
%% cyk.tex -- main
%% ----------------------------------------------------------------

\documentclass[10pt]{article}
\usepackage{amsmath}
\usepackage{amsfonts}
\usepackage{mathtools}
\usepackage[dvipsnames]{xcolor}
\usepackage{subcaption}
\usepackage{graphicx}
\usepackage{algorithm}
\usepackage[noend]{algpseudocode}
\floatname{algorithm}{Algorithm}
\usepackage{enumitem}
\usepackage{tikz}
\usepackage{forest}
\usetikzlibrary{shapes,positioning}
\usepackage{hyperref}
\usepackage[capitalise, noabbrev]{cleveref}
\newcommand{\todo}[1]{{\color{red}#1}}

% Bibliography stuff
\usepackage[
    backend=biber,  
    style=ieee,
    sorting=ynt
]{biblatex}
\addbibresource{main.bib}


%% ----------------------------------------------------------------

\title{Evaluation of the Cocke-Younger-Kasami Algorithm}
\author{Salome Müller}
\date{\today}

\begin{document}

%% Make the title
\maketitle

%% Write the abstract
\begin{abstract}
    The Cocke Younger Kasami algorithm can solve the membership problem for context free grammars in $O(n^3)$.
    For this report three different approaches to the algorithm are implemented and evaluated.
    The evaluation shows that the bottom-up approach is faster on average, while top-down has very fast running times in best-case.
    Further, bottom-up is specialized to operate on linear grammars, yielding running times in $O(n^2)$.
    The bottom-up approach can also be used to calculate the number of errors and correct the input string, such that the result belongs to the language of the grammar.
    While the correction of the string has linear running time and proceeds very fast, the generalized bottom-up algorithm is much slower than the original bottom-up algorithm.
\end{abstract}

\pagebreak
\tableofcontents
\pagebreak

%% Include all chapters
\section{Introduction}

The question whether a given string is in the language of a given context free grammar (CFG) is known as the membership problem.
A CFG contains a set of variables and productions.
Applying the productions on the variables derives a finite set of strings, which form the language of the grammar.
Na\"{i}ve approaches to solve the membership problem have exponential running time.
In this report, we evaluate the Cocke-Younger-Kasami algorithm, which solves the problem in $O(n^3)$ for grammars that are in Chomsky normal form.
We look at a bottom-up approach and a top-down approach which uses memoization.
Further, we try to specialize the algorithm to parse strings for linear grammars.

The evaluation shows that the bottom-up and top-down algorithms behave very differently.
The running times of bottom-up are steadier than those of top-down, i.e., the running time varies less for parsing different strings of the same length for the same grammar.
We further see that the bottom-up approach behaves better on average.
Nevertheless, the top-down approach can yield a lot faster running times, depending on the order of the productions of the grammar and the input string.

However, adapting the bottom-up algorithm to parse linear grammars yields even better results.
The running time for the specialized algorithm lies in $O(n^2)$.

The CYK algorithm can further be used to not only solve the membership problem, but to also compute the minimal number of errors in the input string.
This is the minimal number of symbols in the input string that must be replaced or deleted for the string to be in the language of the grammar.
This algorithm is in $O(n^3)$ too, but in practice it is slow compared to the non-generalized algorithm.
In a second step, the error count can be used to correct the string. The corresponding algorithm is in $O(n)$.

In~\cref{sec:backgrund} we define context-free grammars and introduce the CYK algorithm, as well as the three different approaches used for the evaluation.
In~\cref{sec:specialization} we show how we specialize the algorithm for linear grammars and how the CYK-algorithm can be used to detect and correct errors.
Then, we evaluate all algorithms in~\cref{sec:Evaluation}.



\section{Background}

The Cocke-Younger-Kasami algorithm, which we analyse in this report, solves the membership problem for context free grammars.
In this section we show what a context-free grammar is and how the algorithm operates on it.
Further, we introduce the three approaches for implementing the algorithm which were used in this evaluation.
\todo{And whatever we do in step 3}

\subsection{Context-Free Grammar}

Context-free grammars (CFG) can be used to formalize different types of languages.
They can, for example, be used in computer science, to define the structure of programming languages, or in linguistics to define the structure of any language.

A CFG contains a set of rules, also called productions.
Starting from a certain variable, this productions can be applied to get a sequence of terminal symbols, for example a sentence of the English language.
The sequences that can be generated with a CFG build a language, the grammars context-free language (CFL).

Formally, we define a CFG $G$ by the 4-tuple $G=(V,\Sigma,R,S)$, where $V$ are all non-terminal, and $\Sigma$ all terminal symbols.
$R$ is the set of productions and $S\in V$ is the start symbol.

The productions are of the form $A\rightarrow\alpha$, where $A$ is a variable in $V$ and $\alpha$ is a string of symbols from $(V\cup T)^*$.
If $R$ contains multiple rules for one non-terminal we abbreviate these rules as $A\rightarrow \alpha_1 | \alpha_2 | \dots |\alpha_k $, where $\alpha_i$, $i\in {1\dots k}$ is the right hand side of one of the rules for non-terminal $A$.

If for any strings $u,v\in (V\cup \Sigma^*_)$ there is a production which transforms $u$ to $v$, we say $u$ directly yields $v$, denoted as $u\Rightarrow v$.
If $v$ can be reached by applying multiple productions on $u$ we say $u$ yields $v$, denoted as $u\Rightarrow^* v$, i.e., there is a set of strings $u_1, u_2, \dots, u_k\in (V\cup\Sigma)^*)$ such that $u\Rightarrow u_1 \Rightarrow u_2 \Rightarrow \dots \Rightarrow u_k \Rightarrow v$.

The language of $G$, $L(G)$, contains all strings that can be yielded from $S$, i.e. $L(G)={w\in \Sigma^*: S\Rightarrow^{*}w}$.

The membership problem, which is solved by the CYK-algorithm, is the problem of determining whether a given string is in the language of a grammar.

The following subsections show one simple example of a context-free grammar and introduce different forms of grammars.

\subsubsection{Exemplary Grammar}
\label{subsec:exemplary_grammar}


One simple example is the grammar, whose language consists of all words of the form $(a^n b^n)$, for any $n\in \mathbb{N}$.
This grammar can be defined as $G = ({S, A, B}, {a, b}, R, S)$, where $R$ contains the following rules: 

\begin{align*}
S&\rightarrow ASB \\
A&\rightarrow a \\
B&\rightarrow b \\
\end{align*}

Any string of length $2n$ can be generated by applying $S\rightarrow ASB$ $n$ times, and then replacing all non-terminal $A$ and $B$ with $a$ and $b$ respectively.


\subsubsection{Chomsky Normal Form}

Every context-free grammar can be transformed into an equivalent representation in Chomsky normal form (CNF).
Two grammars are considered equivalent if they generate the same language.
A grammar is in CNF, if all its productions are of the form:

\hspace*{0.5cm}$A\rightarrow BC$\\
\hspace*{1cm}$A\rightarrow a$\\
\hspace*{1cm}$S\rightarrow \epsilon$\\

While $S$ is the start symbol, $A$, $B$ and $C$ are any non-terminal variables, but neither $B$ nor $C$ may be the start variable.
$a$ is a terminal variable.
The start symbol is the only variable, which may yield the empty string, provided the empty string is part of the language.
Further, a non-terminal must either yield two non terminals or one terminal variable.

In order to transform the grammar from~\ref{subsec:exemplary_grammar} to CNF, we add the non-terminals $C,D$ to $V$ and get the new set of productions $R'$:
\begin{align*}
    S&\rightarrow AB | AC \\
    C&\rightarrow DB \\
    D&\rightarrow AC|AB \\
    A&\rightarrow a \\
    B&\rightarrow b \\
\end{align*}

The CYK algorithm can only operate on grammars, that are in the (reduced) Chomsky normal form.
The reduced CNF is similar to CNF, with the only difference that $S$ may also appear on the right hand side of a production.
The following subsection describes how the CYK-algorithm operates.

\subsubsection{Linear Grammars}
\todo{define by using the book.}

\subsection{Cocke-Younger-Kasami Algorithm}

The Cocke-Younger-Kasami algorithm (CYK) solves the membership problem.
For a given Grammar $G$ and an input string $s[1..n]$, it returns the truth-value of whether the $s$ is in the $G$.
For the original algorithm, $G$ should be in reduced CNF or CNF.

The algorithm solves the membership problem in a bottom up manner.
It maintains a table $tab$ of size $n\times n$, where $tab[i,j]$ contains all non-terminals that can yield the substring of $s$ of length $j$ starting at position $i$.

First, $tab$ is initialized as an empty $n\times n$-matrix.
The algorithm then starts by looking over all substrings of size 1, to find the non-terminals that produce the terminals of the input string.
When this is done, $tab[i,1]$ contains ${A: A\in V and A\leftarrow s[i] \in R}$ for all $i\in {1\dots n}$.
The algorithms continues by increasing $j$, starting at $2$ till $n$, and iterating over all possible $i$, $1\leq i < n-j$.
Since the algorithm proceeds bottom-up, when looking at a substring of length $j$, the solution for all strings of length $j-1$ is already known.
Thus, to deduce whether non-terminal $A\in V$ can yield a substring $s[i...i+j]$ the algorithm iterates over all non-terminal productions of $A$.
For each rule $A\rightarrow BC$ of $A$ in $R$ it uses the solutions of former solved subproblems to find whether $B$ and $C$ yield them.
If for any splitting point $k$ of the current substring, $0 < k < j$, $B$ yields the left part of the split substring, $B\Rightarrow s[i..i+k]$, and $C$ yields the left part, $B\Rightarrow s[i+k+1..i+j]$, then $A\xRightarrow{*} s[i..i+j]$.

For each $tab[i,j]$ with $2\leq j \leq n$ and $1\leq i \le n-j$ the algorithm iterates over all non-terminals $A\in V$ and their productions $A\rightarrow BC \in R$.
For each rule, it iterates over all possible splitting points $k$, $1\leq k\le j$.
If $B$ is in $tab[i, k]$ and $C$ in $tab[i+k,j-k]$, then $A$ is added to $tab[i,j]$.

The technique of dividing the problem into smaller subproblems and use their solutions to solve the problem is called Dynamic programming.

\subsubsection{Dynamic Programming}
\todo{introduce dynammic programming, say how it works.}

CYK applies the dynamic programming technique by dividing the problem into two smaller subproblems and solving them each respectively.

In order for dynamic programming to be applicable on a problem, the problem must fulfil two requirements; optimal substructure and overlapping subproblems.
Optimal substructure says, that the optimal solution of a problem can be build from the optimal solutions of a set of subproblem. 
The membership problem looks for a truth-value, the optimal value is therefore true.
When trying to find the truth-value for a non-terminal variable $A$ for any string, we thus try to find any combination of a splitting point $k$ and a production in $R$ with $A$ on its left-hand side, for which both subproblems (the left and right substring), are optimal, i.e., true.
If we find such a combination, we can use the answers of both subproblem; the answer of our problem is the logical \textit{and} of the solution of both subproblems.

To proof that the problem has overlapping subproblems, we will give a small example.
Assume we run the algorithm on an input string of length 6, $s[1..6]$, and our grammar contains a non-terminal variable $A$ with $(A\rightarrow BC), (A\rightarrow CB)\in R$.
At some point, the algorithm might check if $A$ can yield certain substrings of length 4, i.e., substrings $s[1..4]$ and $s[2..6]$.
When considering the first rule, $(A\rightarrow BC)\in R$, on $s[1..4]$, it will check whether $B$ yields $s[1..2]$ and $C$ yields $s[3..4]$.
When considering the second rule, $(A\rightarrow CB)\in R$, on $s[2..6]$, it will check whether $C$ yields $s[3..4]$ and $C$ yields $s[5..6]$.
Finding the truth-value for $C$ on $s[3..4]$ is a subproblem, which will be solved twice, and therefore overlapping.
\todo{draw a small tree to visualize this}

The CYK algorithm has a very good worst-case running time of $O(n^3 * |G|)$.
In practice there are algorithms with a better average case running times.
In the following subsections, we go into more detail on the running time, while introducing three different parsing algorithms, which were used for the evaluation.
The first one is a naive approach, the second one the original CYK-algorithm, and the third one a top-down approach, which makes the naive approach more efficient by introducing memoization.



\subsubsection{Naive}

The naive approach is a recursive depth-first implementation.
It does not use dynamic programming, therefore each subproblem may get solved multiple times.
The input string will be stored globally, as an array of characters $s[1..n]$.
The procedure takes three input arguments, a non-terminal $A$ and the starting and end points of the substring, $i$ and $j$, which should be considered.
If it is called on a substring of length 1, i.e., $i = j-1$, it returns the truth-value of whether $A\rightarrow s[i]$ holds.
This is done in lines 1 to 7 in~\ref{alg:naive}.
Otherwise, it iterates over all non-terminal rules of $A$, $(A\rightarrow BC) \in R$, trying to find a splitting point $k$, $i <= k < j$, for which $B$ yields $s[i..k]$ and $C$ yields $s[k+1..j]$.
If no such rule can be found, the algorithm returns false, since $A$ can not yield $s[i..j]$.
The initial call on the method is \texttt{Naive($S$, $0$, $n$)}.

\begin{algorithm}[H]
    \caption{Naive Parser}
    \label{alg:naive}
    \begin{algorithmic}[1]
        \Procedure{Naive}{non-terminal A, int i, int j}
        \If{$i = j-1$}
            \If{$(A\rightarrow s[i])\in R$}
                \State \textbf{return} true
            \Else
                \State \textbf{return} false
            \EndIf
        \EndIf

        \For{$(A\rightarrow BC) \in R$}
            \For{$k \in \{i+1,\dots,j-1\}$}
                \If {\Call{Naive}{$B$, $i$, $k$} \textbf{and} \Call{Naive}{$C$, $k+1$, $j$}}
                    \State \textbf{return} true
                \EndIf
            \EndFor
        \EndFor

        \State \textbf{return} false
        \EndProcedure
    \end{algorithmic}
\end{algorithm}

The complexity of this algorithm is exponential in $n$, the length of the input string\todo{add reference to book}.
We expect this approach to be the slowest of the three.


\subsubsection{Bottom-Up}
This is the original CYK-algorithm.
It initializes an empty table $tab$ of size $|V|\times n\times n$.
When the algorithm is finished, $tab[A,i,j]$ will be true, if non-terminal $A$ can yield $s[i..i+j]$.
Notice that $j$ is no longer the end point of the substring, but its length.

Since the algorithm performs bottom-up, it first fills the bottom row of the table, i.e. $tab[A,i,1]$ for $i\in{1,\dots,n}$ and all $A\in V$.
It then iteratively increases $j$, and fills all cells on its way up through the table.
At each cell $tab[A,i,j]$, the algorithm checks if there is a rule $A\rightarrow BC$ in $R$ and a $k$, $1 \leq k < j$, for which $tab[B,i,k]$ and $tab[C,i+k+1, j]$ are both true.
This way, it uses the solutions to already solved subproblems to solve the current problem, only accessing cells of $tab$, which were already filled before.
If so, $A$ can yield $s[i..i+j]$, and $tab[A,i,j]$ will is set to true.
Since the algorithms were implemented in Java, $tab[C,i+k+1, j]$ will only be accessed if $tab[B,i,k]$ is true.


\begin{algorithm}[H]
    \caption{Bottom-Up CYK Parser}
    \label{alg:bu}
    \begin{algorithmic}[1]
        \Procedure{Bottom-Up}{input string $s[1..n]$}
        \State allocate table $tab[|V|][n][n]$ initialized with false
    
        \For{$i \in {1,\dots,n}$}
            \For{${A: A\rightarrow s[i]\in R}$}
                \State $tab[A,i,1] \leftarrow$ true
            \EndFor
        \EndFor

        \For{$j\in {2,\dots,n}$} \hspace*{2.75cm}\textit{-- length of substring}
            \For{$i\in {1,\dots,n-j+1}$} \hspace*{1cm}\textit{-- starting point of substring}
                \For{$(A\rightarrow BC) \in R$} \hspace*{1cm}\textit{-- all productions}
                    \For{$k \in {1,\dots,j-1}$} \hspace*{0.5cm}\textit{-- all splitting points}
                        \If {$tab[B,i,k]$ \textbf{and} $tab[C,i+k,j-k]$}
                            \State $tab[A,i,j]\leftarrow$ true
                            \State break loop
                        \EndIf
                    \EndFor
                \EndFor
            \EndFor
        \EndFor

        \State \textbf{return} $tab[S,1,n]$
        \EndProcedure
    \end{algorithmic}
\end{algorithm}

The bottom-up CYK algorithm solves each subproblem exactly once.
It has a complexity of $O(n^3)$\todo{reference book}.
The initialization, lines 3 to 5, takes $O(n)$, due to the iteration over all elements of $s$.
The for-loop on line 9 is repeated at most $n$ times since $k$ is in the interval $\{1,\dots,n\}$ at most (in practice, it will often be executed less, since it breaks, as soon as the condition is met).
The two outer loops, lines 6 and 7, are both repeated $n$ times.
Thus, the loop at line 9 will be called $O(n^2)$ times, which results in a complexity of $O(n^3)$ for lines 6 to 12.
The overall running time is therefore in $O(n^3)$.

We expect this algorithm to behave very similarly on strings of the same length.
Further, the order in which the rules of the grammar are provided does have an impact, but should not affect the running time as much as it does for the top down approach, which we show next.

\subsubsection{Top-Down}
\label{sec:top_down}
The top-down approach resembles the naive one, as it is recursive.
It uses, however, memoization, which makes it a lot more efficient, as each subproblem is solved once at most.
When the method \texttt{Top-Down-Parse(input string $s[1..n]$)}(Algorithm~\ref{alg:td1}) is called, it initializes the global table of size $|v|\times n\times n$, which is similar to the one used for the bottom-up CYK algorithm.
It then calls \texttt{Top-Down($S$, $1$, $n$)}(Algorithm~\ref{alg:td}) and returns $tab[S,1,n]$, which contains the truth value of the membership problem.
\texttt{Top-Down($A$, $i$, $j$)} first checks, whether the subproblem of whether $A$ yields $s[i..j]$ was already solved, i.e., if $tab[A,i,j]$ is set.
If so, it returns the before computed truth-value.
Otherwise, the value is computed recursively, stored in $tab[A,i,j]$ and returned.
The next call of \texttt{Top-Down($A$, $i$, $j$)} will not compute anything, but return the truth-value immediately.

Similar to bottom-up, the right-hand side of the if-request on line 11 will only be called, if the left-hand side is true.

\begin{algorithm}[H]
    \caption{Top-Down Parser}
    \label{alg:td1}
    \begin{algorithmic}[1]
        \Procedure{Top-Down-Parse}{input string $s[1..n]$}
        \State allocate global table $tab[|V|][n][n]$ initialized with null
        \State \Call{Top-Down-Parser}{$S$, $1$, $n$}
        \State \textbf{return} $tab[S,1,n]$
        \EndProcedure
    \end{algorithmic}
\end{algorithm}

\begin{algorithm}[H]
    \caption{Top-Down}
    \label{alg:td}
    \begin{algorithmic}[1]
        \Procedure{Top-Down}{non-terminal A, int i, int j}
            \If{$tab[A,i.j]=$\texttt{null}}
                \State \textbf{return} $tab[A,i,j]$
            \EndIf

            \State $tab[A,i.j]\leftarrow$ false

            \If{$j = 0$}
                \If{$(A\rightarrow s[i])\in R$}
                    \State $tab[A,i.j]\leftarrow$ true
                \EndIf
            \Else
                \For{$(A\rightarrow BC) \in R$}
                    \For{$k \in \{i+1,\dots,j-1\}$}
                        \If {\Call{Top-Down}{$B$, $i$, $k$} \textbf{and} \Call{Top-Down}{$C$, $i+k$, $j-k$}}
                        \State $tab[A,i.j]\leftarrow$ true
                        \State break loop
                        \EndIf
                    \EndFor
                \EndFor
            \EndIf

            \State \textbf{return} $tab[A,i.j]$
        \EndProcedure
    \end{algorithmic}
\end{algorithm}

The complexity of this algorithm is, similar to the bottom-up algorithm, $O(n^3)$.
The difference between the two is, that bottom-up fills all cells of $tab$, while top-down only fills the ones it passes while trying to find a solution.
In practice, its running time is therefore more dependant on the input string itself, as well as the grammar.
Depending of the order of the rules, it may yield very different running times.
If it consults rules, that yield the considered substrings in the beginning, it does not consult other rules, and must therefore solve a lot less subproblems.
If this is not the case, and most of the subproblems must be solved, then we expect the algorithm to be slower than bottom-up.

\subsubsection{Implementation}
Write a little bit about what data structures where used to represent the productions etc. Necessary?


% \todo{The parser was implemented in Java.
% It can be called with instructions on the set on input string, also called test set, and the algorithm which should be used.

% The rules of the grammar were implemented in a way, that the algorithms can easily access all rules of a variable and iterate over them.
% The non-terminal rules of each variable in $V$ are stored in an array of arrays of integers.
% The arrays of each variable are referenced from an array \texttt{rules}.
% If the start variable has one rule, say $v_0 \rightarrow v_1v_2$, then \texttt{rules[0][0] = [1, 2]}

% The terminal rules are stored twice, once as rules of the terminal symbols and once as the rules of the non-terminal variables, making both [reference algorithms] efficient.
% }



\section{Generalization and Specialization}
\label{sec:specialization}

In this section we try a specialization and a generalization of the CYK algorithm.
For the specialization we parse grammars in a different form, namely \textbf{linear grammars}, instead of grammars in CNF.
We first convert these grammars to CNF and compare the running times to the previous experiments.
In a second step, we adapt the CYK algorithm to parse strings for those grammars, and compare the efficiency of both approaches.

The generalization extends the bottom-up CYL algorithm by not only returning truth values for the membership problem, but by computing the errors in the input string.
This error count is then used to correct the input string, such that it becomes a member of the language.

\subsection{Specialization with Linear Grammars}
Similar to CNF, linear grammars also have certain restrictions on how the productions may look like.
They have, however, only one restriction; each production may at most have one non-terminal variable on its right-hand side.

We use \textbf{linear context free-grammars in Chomsky normal form} to generalize the CYK algorithm.
These are grammars that are linear and where all productions have either one terminal symbol, or a non-terminal variable and a terminal symbol on their right-hand side.
The example we gave in~\cref{subsec:exemplary_grammar}~can be easily transformed into linear CNF, by removing non-terminal variable $A$, resulting in the following productions:
\begin{align*}
    S&\rightarrow aB \\
    B&\rightarrow Sb|b \\
\end{align*}


\subsubsection{Transform Linear grammars to CNF}
A linear grammar can be easily transformed into CNF, by introducing a non-terminal variable $a_T$ for each terminal symbol $a$ which appears in a non-terminal production.
For every variable that is added that way, the terminal production $a_T\rightarrow a$ is added to the set of productions.
For the exemplary grammar this would give the productions
\begin{align*}
    S&\rightarrow a_T B \\
    B&\rightarrow Sb_T|b \\
    a_T&\rightarrow a \\
    b_T&\rightarrow b
\end{align*}

By adding non-terminal variables to the grammar, we expand one dimension of the memoization table.
Since those have no non-terminal productions, for both the top-down and bottom-up algorithm no additional rules are tested.
However, the cells for those non-terminals may be accessed often for strings of length bigger than 1, always returning false.
These unnecessary calls may extend the running time.
We thus expect test runs on this grammar to yield similar running times than for equivalent grammars already in CNF.
If we do not transform the grammar into CNF but instead adapt the CYK-algorithm, this extension may be avoided.

\subsubsection{Adapt CYK to Linear Grammars}
Grammars in Linear CNF have two types of rules; terminal rules and non terminal rules.
Similar to the non-specialized approach, we can use the terminal rules to determine, which non-terminal variables can directly derive which terminal symbols.
The non-terminal rules are different, and require the algorithm to act differently.

They all have exactly on terminal and one non-terminal variable on their right hand side.
In contrast to the non-specialized approaches, we do not need to look at multiple splitting points, to determine whether a non-terminal rule applied on a given non-terminal variable can derive a string.
The terminal symbol must be equal to the last or first symbol of the substring, depending on wether it is the first or second variable on the right-hand side of the production.
The non-terminal variable must be able to derive the remaining symbols of the considered string.
Therefore, only the last or the first splitting point need to be considered.
\cref{alg:bu_linear}~shows how this can be applied to the bottom-up CYK algorithm.

We chose to adapt the bottom-up algorithm rather than top-down, because it is less likely to run into stack-overflow errors.
For big input strings, top-down may cause them because the compiler loses track of the recursive calls.

\begin{algorithm}[H]
    \caption{Linear Bottom-Up CYK Parser}
    \label{alg:bu_linear}
    \begin{algorithmic}[1]
        \Function{Lin-Bottom-Up}{input string $s[1..n]$}
        \State allocate table $tab[|V|][n][n]$ initialized with false
        \State counter $\leftarrow$ 0
    
        \For{$i \in {1,\dots,n}$}
            \For{${A: A\rightarrow s[i]\in P}$}
                \State $tab[A,i,1] \leftarrow$ true
            \EndFor
        \EndFor

        \For{$j\in {2,\dots,n}$}
            \For{$i\in {1,\dots,n-j+1}$}
                \For{$(A\rightarrow v_1v_2) \in P$}
                    \State counter $\leftarrow$ counter + 1 \label{lst:bu_linear_11}
                    \If {$v_1$ is terminal symbol}
                        \If {$v_1 = s[i]$ \textbf{and} $tab[v_2,i+1,j-1]$}
                            \State $tab[A,i,j]\leftarrow$ true
                            \State break loop
                        \EndIf
                    \Else
                        \If {$v_2 = s[i+j]$ \textbf{and} $tab[v_1,i,j-1]$}
                            \State $tab[A,i,j]\leftarrow$ true
                            \State break loop
                        \EndIf
                    \EndIf
                \EndFor
            \EndFor
        \EndFor

        \State \textbf{return} $tab[S,1,n]$
        \EndFunction
    \end{algorithmic}
\end{algorithm}

The only difference between~\cref{alg:bu} and~\cref{alg:bu_linear} is what happens at the inner most loop, beneath~\cref{lst:bu_linear_11}.
Instead of iterating over all possible splitting points, we first check which part of the rule's right-hand side is the terminal symbol.
We then ask if this is equal to the respective symbol in the input string.
If so check whether the terminal variable can derive the rest of the substring.

The running time of this algorithm is $O(n^2)$, since it is similar to the non-specialized bottom-up algorithm, but the inner most for-loop is not $O(n)$ anymore, but constant.

Compared to parsing strings for a linear grammar, which was transformed to CNF, we expect this algorithm to be more efficient, since it must try less splitting points per production.
Further, it will use less memory, since there are less non-terminal variables, compared to the transformed grammar, thus one dimension of $tab$ is smaller.

\pagebreak
\subsection{Generalization for Error Correction}
One may not only be interested in whether the input string is in the language, but also in how many errors there are and how the string could be altered to fit into the language.
The bottom-up CYK algorithm can be adapted to fit this purpose.
\cref{alg:bu_err} resembles the bottom-up algorithm from~\cref{sec:bottom_up} but generalizes it by counting the errors.
For this, every cell of the memoization table will contain the configuration, i.e., the rule and splitting point, that derive a string which is closest to the actual substring of the input string.
We chose the bottom-up approach, since all cells of the table must be filled to find this minimal configuration, and the bottom-up algorithm performs better on problems where all cells are filled, compared to the top-down algorithm(\cref{sec:Evaluation}).

An error is a symbol in the input string which must be either deleted or replaced with another terminal symbol, in order for the start-variable to be able to derive the string.
To count these deletions and replacements, the structure of the memoization table $tab$ is altered.
$tab$ no longer stores boolean values, but 4-tuples, i.e., $(e, B, C, k)$, for every $tab[A,i,j]$.
Among all strings which $A$ derives, the closest ones to substring $s[i..i+j]$ of the input string are the ones with $e$ errors.
This means that $e$ symbols must either be replaced or deleted from $s[i..i+j]$.
This string is derived by applying the rule $A\rightarrow BC$, where $B$ and $C$ derive the left and right part of $k$, respectively.

In a second step, the memoization table can then be used to build a corresponding string that is in the language of the grammar.
This is done by applying the productions as they are stored in $tab$, starting at $tab[S,1,n]$, and deleting or replacing a symbol when necessary.

\subsubsection{Counting the errors}
\cref{alg:bu_err} is the generalized bottom-up algorithm.
The generalized algorithm requires an additional step for the initialization.
The loop at~\cref{lst:bu_err_ini_1} sets the bottom row of $tab$ for each non-terminal variable.
The value is either set to 1, if the variable has a terminal rule, or to $n$ otherwise.
The second initializing loop starts at~\cref{lst:bu_err_ini_2} and resembles the one from the initial bottom-up algorithm, with the only difference that it sets the value of cells to $0$ instead of \textit{true}.

The algorithm then proceeds in a very similar manner as the original bottom-up algorithm.
The only other difference is that instead of looking for a rule and splitting point for which both subproblems yield \textit{true}, we iterate over all subproblems, picking the one with a minimal count of errors $e$.
For this, we compare the value in $tab$ with the sum of the value of the two subproblems (\cref{lst:bu_err_comp}).
If the new value is smaller, we memorize the new error value, as well as the configuration that leads to this error count.

If there is a way to transform the string with only replacements into a string of the language, then the algorithm will return the minimal amount of those operations that must be performed.
It always chooses the production which leads to the string with the smallest amount of errors.
When looking at substrings of length 2 and a non-terminal $A$, the algorithm iterates over all rules for $A$, i.e., $A\rightarrow BC$.
There is only one splitting point ($k=1$), and the corresponding entries in $tab$ can only have three different values: $n$, if the non-terminal has no terminal productions, 0, if the variable directly derives the symbol, or 1 if it does not.
Out of all productions of $A$, the algorithm will choose the one where the sum of the two values for $B$ and $C$ are minimal.

If a non-terminal variable has no terminal rules, we call it a dead-end, as it can not derive any substring of length 1.
Such variables have error count $n$ for substrings of length 1 (\cref{lst:bu_err_ini_1}).
Since all cells are initialized with error count $n$, solutions leading to a dead end are ignored due to the comparison at~\cref{lst:bu_err_comp}.
This is important, since these variables can not be used to replace a symbol.

When looking at longer substrings, the algorithm will choose the configurations, that use the former solutions with the smallest amount of errors.
Thus, when the algorithm terminates, each cell will hold the configuration to reach the string closest to the input string in terms of errors.

We excluded deletions from this reasoning, since they are not always computed correctly by this algorithm.
The value returned by the algorithm is however an upper bound on the sum of corrections and deletions that must be performed.

\begin{algorithm}[H]
    \caption{Bottom-Up CYK Parser with Error Count}
    \label{alg:bu_err}
    \begin{algorithmic}[1]
        \Function{Bottom-Up-Error}{input string $s[1..n]$}
        \State allocate table $tab[|V|][n][n]$ initialized with \{$n$, null, null, null\}
        \State counter $\leftarrow$ 0 \label{lst:bu_err_1}
    
        \For{$A\in V$} \label{lst:bu_err_ini_1}
            \State $e\leftarrow n$ 
            \If{$A$ has terminal rules}
                \State $e \leftarrow 1$
            \EndIf
            \For{$i \in \{1,\dots,n\}$}
                \State $tab[A,i,1].e \leftarrow e$
            \EndFor
        \EndFor
        \State
        
        \For{$i \in \{1,\dots,n\}$} \label{lst:bu_err_ini_2}
            \For{${A: A\rightarrow s[i]\in P}$}
                \State $tab[A,i,1].e \leftarrow 0$ \label{lst:bu_err_ini_22}
            \EndFor
        \EndFor
        \State

        \For{$j\in \{2,\dots,n\}$} \hspace*{2.75cm}\textit{-- length of substring}\label{lst:bu_err_loop}
            \For{$i\in \{1,\dots,n-j+1\}$} \hspace*{1cm}\textit{-- starting point of substring}
                \For{$(A\rightarrow BC) \in P$} \hspace*{1.4cm}\textit{-- All productions}
                    \For{$k \in \{1,\dots,j-1\}$} \hspace*{0.5cm}\textit{-- all splitting points}
                        \State counter $\leftarrow$ counter + 1
                        \State $e \leftarrow tab[B,i,k].e + tab[C,i+k+1,j-k-1].e$

                        \If {$e < tab[A,i,j].e$} \label{lst:bu_err_comp}
                            \State $tab[A,i,j]\leftarrow \{e, B, C, k\}$
                        \EndIf
                    \EndFor
                \EndFor
            \EndFor
        \EndFor

        \State \textbf{return} $tab[S,1,n]$
        \EndFunction
    \end{algorithmic}
\end{algorithm}


The two initialization loops (\cref{lst:bu_err_ini_1} to~\ref{lst:bu_err_ini_22})are both in $O(n)$, thus, the initialization is in $O(n)$.
The nested loops starting at~\cref{lst:bu_err_loop} are in $O(n^3)$, as the loops are the same as in the original bottom-up algorithm.
The running time is thus in $O(n^3)$.

However, we expect this algorithm to be slower than the original bottom-up algorithm.
In some cases, each splitting point may yield a slightly smaller error than the previous one, leading to the algorithm updating $tab[A,i,j]$ multiple times.
This is more involved than simply setting a boolean value once \textit{true} is found.
Further, since we are no longer working with booleans, the cells for both non-terminals on the right hand side of the rule are accessed at each iteration.
Thus, the algorithm accesses more cells and updates more objects, leading to a longer running time.
Further, since an array of size four is stored in every cell, where the original algorithm only stored a boolean value, more memory is used for counting the errors.

\subsubsection{Correcting the Input String}
\label{subsec:correcting}
The information stored in $tab$ is used to transform the input string into a word of the grammars language.
\cref{alg:ec} resembles the top-down approach in some ways.
The function \texttt{Correct-String} initializes two global variables: the \texttt{counter} holding the number of recursive calls, and \texttt{errors}, the number of corrected and deleted symbols.
The detected errors are counted again, since~\cref{alg:bu_err} only gives an upper bound.
\cref{alg:ec} increases the variable whenever a symbol is deleted or replaced, and thus returns the actual number of necessary operations.

\texttt{Correct-String} calls \texttt{Correct} on the start variable $S$, the first position of the string (1) and its length $n$.
The input string and $tab$ are both global, so \texttt{Correct} can operate on them.
\texttt{Correct} is a recursive function and takes non-terminal variable $A$, the starting point $i$ and length $j$ of a substring of the input string as parameters. 
The algorithm distinguishes four different cases:
\begin{enumerate}
    \item \textbf{Correct:} If $tab[A,i,j.e]=0$ then the substring is correct and can be returned directly (\cref{lst:ec_correct}).
    \item \textbf{Replacement:} If $j=0$, then $s[i]$ is wrong and replaced with a symbol that can be derived by $A$, i.e. any $a\in \Sigma$ such that $A\rightarrow a \in P$. If $A$ has no terminal rule ($A$ is a dead end), the input string cannot be transformed into a word of the language (\cref{lst:ec_replace}).
    \item \textbf{Deletion:} If a lower error count can be reached by deleting either the first or last symbol of the current substring, then the symbol is deleted:
        \begin{enumerate}[label=\alph*)]
            \item Delete first symbol: if the error count of $tab[A,i+1,j-1]$ is smaller by more than one compared to $tab[A,i,j].e$, then implicitly delete $s[i]$ by returning \texttt{Correct($A, i+1, j-1$)} (\cref{lst:ec_first_delete}).
            \item Delete last symbol: if the error count of $tab[A,i,j-1]$ is smaller by more than one compared to $tab[A,i,j].e$, then implicitly delete $s[i+j]$ by returning \texttt{Correct($A, i, j-1$)} (\cref{lst:ec_last_delete}).
        \end{enumerate}

    \item \textbf{Otherwise:} If none of the above hold, apply the rule stored in $tab[A,i,j]$ and concatenate the corrected version of the corresponding two substrings (\cref{lst:ec_return}).
\end{enumerate} 

\begin{algorithm}[H]
    \caption{Error Correction}
    \label{alg:ec}
    \begin{algorithmic}[1]
        \Function{Correct-String}{}
        \State counter $\leftarrow$ 0
        \State errors $\leftarrow$ 0
        \State \textbf{return} \Call{Correct}{$S$, 1, n}
        \EndFunction

        \Function{Correct}{non-terminal A, int i, int j}
            \State counter $\leftarrow$ counter + 1

            \If{$tab[A,i,j].e = 0$}
                \State \textbf{return} $s[i, i + j + 1]$ \hspace*{2.8cm}\textit{-- substring is correct} \label{lst:ec_correct}
            \EndIf
            \State

            \If{j = 0}
                \State \textbf{return} $c:A\rightarrow c\in R$ \hspace*{2.7cm}\textit{-- replace terminal} \label{lst:ec_replace}
            \EndIf
            \State

            \State first $\leftarrow tab[A,i+1,j-1].e$ \label{lst:ec_first}
            \State last $\leftarrow tab[A,i,j-1].e$ \label{lst:ec_last}
            \If{first $<$ last}
                \If{first $< tab[A,i,j].e - 1$} \hspace*{1.3cm}\textit{-- delete first terminal} \label{lst:ec_first_delete}
                    \State error $\leftarrow$ error $+ 1$ 
                    \State \textbf{return} \Call{Correct}{$A, i+1, j-1$}
                \EndIf
            \Else
                \If{last $< tab[A,i,j].e - 1$} \hspace*{1.3cm}\textit{-- delete last terminal} \label{lst:ec_last_delete}
                    \State error $\leftarrow$ error $+ 1$ 
                    \State \textbf{return} \Call{Correct}{$A, i, j-1$}
                \EndIf
            \EndIf
            \State

            \State B $\leftarrow tab[A,i,j].B$
            \State C $\leftarrow tab[A,i,j].C$
            \State k $\leftarrow tab[A,i,j].k$

            \State \textbf{return} \Call{Correct}{$B, i, k$} + \Call{Correct}{$C, i+k+1, j-k-1$} \label{lst:ec_return}
        \EndFunction
    \end{algorithmic}
\end{algorithm}

The running time of this algorithm depends heavily on the amount and position of the errors in the input string.
If the errors are clustered, leaving long error-free substrings, the algorithm will run faster than if the errors are spread evenly.

The worst running time is obtained when no substring of length greater than 1 is correct and if further no symbol must be deleted.
In this case, as long as the substring is not just one symbol,~\cref{lst:ec_return} is executed.
This results in $2n$ recursive calls.
There are two base cases: either the string is correct (\cref{lst:ec_correct}), which in the worst case only occurs if the string has length one, or it has length one but is incorrect (\cref{lst:ec_replace}).
Both cases run in $O(1)$.
Thus, the overall worst case is in $O(n)$, since we perform a $O(1)$ operation $2n$ times.
We exclude deletion of a symbol from this reasoning, since deleting a symbol leads to making only one recursive call instead of two.

In the following section, we evaluate all algorithms that were introduced in the first two sections.

\section{Evaluation}
\label{sec:Evaluation}
In this section we show different experiments that were run on the different approaches and analyse their running times, as well as the number of iterations in the inner most loop or their recursive calls respectively.
First, we give an overview over different grammars, that were used, and explain how we expect the algorithms to behave when parsing input strings on them.
All of them are in (reduced) Chomsky Normal Form.

\subsection{Grammars}
\subsubsection{Dyck Language}
This language consists of all words, that have the correct amount of opening and closing parentheses, i.e., strings of the form '()...()' or '((...))'.
The grammar that builds these words has the rules:
\begin{align*}
    S&\rightarrow SS|LA|LR\\
    A&\rightarrow SR\\
    L&\rightarrow (\\
    R&\rightarrow )\\
\end{align*}

We will run experiments on words of the language, as well as on strings with additional single parentheses, i.e., ')()...()' and ''()...()(', which are not part of the language.

For this grammar, we expect top-down to run faster on strings of the form '()..()'.
The algorithm iterates over the rules in the order as they were parse to the program, thus $S\rightarrow SS$ is the first rule.
It then iterates over different splitting points, starting with $k=1$, which will not find a solution, as $S$ can not yield any of the substrings, since they have a different amount f opening and closing parentheses.
when trying $k=2$, \texttt{Top-Down($S$, $1$, $2$)} is called, which will return true.
For the right hand side of the string, this will be repeated.
Thus, with the first rule of the grammar and the second splitting point, the optimal answer is returned and the algorithm is expected to terminate relatively fast.

For strings of the form '((..))', the top-down algorithm will look at a lot more subproblems while looking for the solution.
Different from strings of the form '()..()' this string has no partitioning into two substrings, where $S$ can yield both substrings.
All rules have to be applied on $S$ at least once, thus a lot more rules and parts of the string will be considered before the answer is found.

Further, ')()..()' is parsed faster than '()..()(' by the top-down algorithm.
The way the algorithm was implemented makes us not look at the right part of a splitting (second call of \texttt{Top-Down} on line 10 in algorithm\ref{alg:td}).
This means, when the first symbol in the string violates the constraints of the language, we look at the left hand side of every partitioning, find that it can not be yielded, and terminate.
If, on the other hand, the last symbol violates the constraint, we will find for a lot of splitting points, that $S$ can yield the left substring, and check the right substring.
This yields in a lot more subproblems which have to be considered, and thus in a longer running time.


\subsubsection{Strings Starting or Ending in a}
This grammar contains all worlds with an arbitrary number of a's and b's in any order, but starting resp. ending with a.
The rules for strings starting with a are:
\begin{align*}
    S&\rightarrow AB\\
    B&\rightarrow BB|a|b\\
    A&\rightarrow a\\
\end{align*}

and those for strings ending in a:
\begin{align*}
    S&\rightarrow BA\\
    B&\rightarrow BB|a|b\\
    A&\rightarrow a\\
\end{align*}

For both of these grammars we will run tests on strings of the form 'ab..ab' and 'ba..ba'.
We expect a similar behavior as in the Dyck grammar.
The bottom-up approach will take a similar amount of time for both sets of test strings with either grammar.
Top-down on the other hand will perform better on the strings that are in the language for both grammars, as well as when parsing the strings not starting in a with the grammar starting in a.
It performs worse when parsing strings that do not end in a with the grammar that ends in a, with the same reasoning as we applied for the Dyck language.
Top down runs faster, when the symbol that violates the constraint of the grammar is in the beginning of the string.


\subsubsection{Equal Numbers}
Equal numbers is a grammar, that yields all strings with the same amount of a's and b's.
This is achieved with the following rules:
\begin{align*}
    S&\rightarrow SS|AB|BA\\
    B&\rightarrow SB|BS|b\\
    A&\rightarrow a\\
\end{align*}

For this grammar we will test strings of the form aa..bb and ab..ab, as well as both of these sets with additional a's.
We expect both parser to run slower on this, than on strings starting with a.
It may be similar to strings ending in a though, since we always must look at the whole string.
An additional a in the end may be worse than one at the beginning.
It is also worse than parentheses on '()..()', but similar to '((..))'.
top-down will be faster on ab..ab than bottom up.
\todo{think more about expectations, very little time invested so far}

\subsubsection{A Finite Language}
\todo{come up with a finite language ?}

\subsection{Experiments}
We will now show the results of experiments, using the grammars described, and analyze whether  the algorithms behave as we expect them to behave.
In order to get better results, for each input string the parser was run 10 times.
The time showed in the plots is the average of the resulting running times, where the fastest and slowest times were excluded.

\subsubsection{Dyck Language}

\begin{figure}[!ht]
    \centering
    \includegraphics[width=0.6\textwidth]{Images/t_dyck_bu.jpg}
    \caption{Running time (s) of the bottom-up algorithm when parsing different set of strings of sizes 100-5000, in steps of hundred, for the Dyck Language.}
    \label{fig:t_dyck_bu}
\end{figure}

Figure~\ref{fig:t_dyck_bu} shows the running times of the bottom-up parser on different sets of input strings.
The strings were of length 100 to 5000, growing in steps of 100.
As we assumed, the times are very similar for all four different sets of input strings.
Parsing the strings of the form '((..))' is a little faster.
We do not know why.

The curves are asymptotically to $O(n^3)$, in fact, the yellow, blue and green line are very close to $6.4*10^{-10}*n^3$.

We split the parsings of top-down into two plots.
The first one, Figure~\ref{fig:t_dyck_td_slow}, is for the slow cases, where we did not run the parser on strings longer than 2500
The second one, Figure~\ref{fig:t_dyck_td_fast}, was for the fast cases, where we extended the test set to contain strings up to a size of 10000.

\begin{figure}[h!]
    \centering
    \includegraphics[width=0.6\textwidth]{Images/t_dyck_td_slow.jpg}
    \caption{Running time (s) of the top-down algorithm when parsing two set of strings of sizes 100-2500, in steps of hundred, for the Dyck Language.}
    \label{fig:t_dyck_td_slow}
\end{figure}

\begin{figure}[h!]
    \centering
    \includegraphics[width=0.6\textwidth]{Images/t_dyck_td_fast.jpg}
    \caption{Running time (ms) of the top-down algorithm when parsing two set of strings of sizes 100-10000, in steps of hundred, for the Dyck Language.}
    \label{fig:t_dyck_td_fast}
\end{figure}

As we assumed, parsing the test sets of strings of the form '((..))' and '()..()(' with top-down was a lot slower than parsing the strings of the other two sets.
While parsing strings of the form '()..()(' of length 2500 took almost 70 seconds, parsing strings of the form '()..()' of length 10'000 took only 0.4 seconds.

The fast cases are a lot faster than the bottom-up parser.
This is the case, because bottom up fills all cells of $tab$, regardless of whether or not they are needed to find the solution, while top down only fills the one needed to find the optimal solution.
When the rules of the grammar are in a favorable order and the splitting points for finding subproblems that yield the optimal solution is low, as it is the case in the fast cases of Dyck, it can be very fast.

However, if this is not the case, then the parser may take a lot of time.
We can see this at the slow cases of the Dyck language.
Here, the parser behaves a lot worse than bottom up.
This may be due to the fact, that bottom up fills the cells of $tab$ in a structured way, accessing the already filled cells of $tab$ not more often, then needed to fill the other cells.
Top-down on the other hand may run into the same subproblems very often.
This means the algorithm calls itself recursively and to access the cell of the same subproblem more often than bottom up does.
Since recursive calls are more time consuming, and more accesses may be performed, this results in a potentially very bad running time.

\begin{figure}[h!]
    \centering
    \includegraphics[width=0.6\textwidth]{Images/t_dyck_order.jpg}
    \caption{Running time (ms) of both algorithms for parsing a set of strings of sizes 100-2500, in steps of hundred, for the Dyck Language, with a different order of the rules.}
    \label{fig:t_dyck_order}
\end{figure}

I order to verify the hypothesis, that the order of rules matters for the top-down parser, we run experiments on the same grammar, but with the rules of $S$ in opposite order.
The results can be seen in figure~\ref{fig:t_dyck_order}.
The parser is in fact a lot slower than it was before, thus the order of the rules may play a major rule, when parsing.
We further see that it does not matter that much for the bottom up parser.
It has almost the same running time, than it had with the original ordering of the rules.

\subsection{Strings starting and ending in a}

The running times achieved, did not meat the expectations.
Surprisingly, both the top down and bottom-up algorithm performed very differently on the two grammars, being very fast at parsing for the grammar for words starting in a, and slower for words ending in a.
In general, we see that the times are lower than they were for the Dyck language.
This is presumably because this grammar has only two non-terminal rules, which leads to less subproblems which have to be considered.

We split the results in three graphs, one for the running times of both parsers on the grammar ending in a (Figure~\ref{fig:t_ea_td_bu}), one for top-down and one for bottom-up, each for the grammar starting in a (Figure~\ref{fig:t_sa_td} and ~\ref{fig:t_sa_bu} respectively).

\begin{figure}[h!]
    \centering
    \includegraphics[width=0.6\textwidth]{Images/t_ea_td_bu.jpg}
    \caption{Running time (ms) of the bottom-up and top-down algorithm when parsing two set of strings of sizes 100-2500, in steps of hundred, for the Language of words ending in a.}
    \label{fig:t_ea_td_bu}
\end{figure}

We see, that the curves for parsing the grammar ending in a are almost identical with the ones of the bottom up parser run on the Dyck Language.

\begin{figure}[h!]
    \centering
    \includegraphics[width=0.6\textwidth]{Images/t_sa_bu.jpg}
    \caption{Running time (ms) of the bottom-up algorithm when parsing two set of strings of sizes 100-2500, in steps of hundred, for the Language of words starting in a.}
    \label{fig:t_sa_bu}
\end{figure}

However, the running times for the grammar for strings starting in a are a lot faster.
If we consider how $tab$ will be filled by the algorithm, it becomes clear, why that is.
Remember that $tab$ for this grammar is $3\times n\times n$, since it has three non-terminal variables.
we will look at the two dimensional table of each non-terminal.

The table for $A$, say $tab_A$, has true only in the row for substrings of length 1 and where the symbol is $a$.
As it has no non-terminal rule, when trying to fill the reminder of $tab_A$ the algorithm proceeds very fast, since no splitting has to be considered, as the corresponding loop over $k$ is never even accessed.

The table for $B$, $tab_B$, has true in all cells.
All substrings of length one can be yielded, since $B$ has the two terminal rules $B\rightarrow a|b$.
Further, since its only non-terminal rule is $B\rightarrow BB$, when trying to fill a new cell of $tab_B$, only other cells of $tab_B$ must be considered.
BEcause they are all true, the loop over the splitting point $k$ gets breaked after the first splitting point, thus filling $tab_B$ can be done in a slow matter, too.

Let's now look at $tab_S$.
Since the only rule for $S$ is $S\rightarrow AB$, first the cell of $tab_A$ gets accessed.
The first splitting point always generates a substring of length one, on the left of k.
If this is an a, then the corresponding cell in $tab_A$ is true, and we must access $tab_B-$ as well.
As we argued before, this value will always be true, we are thus not looking at any other splitting points.
If the substring is b, then the cell in $tab_A$ is false, and we will not look at $tab_B$, but continue to the next splitting point.

For the grammar for all words ending in a, $tab_A$ and $tab_B$ are filled in a very similar manner.
However, when filling $tab_S$ we have almost twice the amount of table accesses to $tab_B$.
Since the rule for $S$ is $S\rightarrow BA$, for every splitting point first $tab_B$ gets accessed, which will always return true, and then $tab_A$ gets accessed, which will return false in most cases.
Thus, both $tab_B$ and $tab_A$ are accessed for every $k$, while for strings starting in a $tab_B$ was only accessed when $tab_A$ was true.

For strings starting in a, we expect very similar running times for strings of the form aa..bb.
For strings ending in a, we expect worse running times for those strings.
We would expect it to be even worse for strings of the form ab..bb, while for strings starting in a it would result in a similar running time.
\todo{run experiments, include plots}


\begin{figure}[h!]
    \centering
    \includegraphics[width=0.6\textwidth]{Images/t_sa_td.jpg}
    \caption{Running time (ms) of the top-down algorithm when parsing two set of strings of sizes 100-5000, in steps of hundred, for the Language of words starting in a.}
    \label{fig:t_sa_td}
\end{figure}

as we see in Figure~\ref{fig:t_sa_td}, the running times for top-down were incredibly low.
The three bumps in the blue line are supposedly due to rounding errors of the compiler, seen as the times there are lower than 5 milliseconds.
The bumps appear on the red line at the same time after the same period of time after starting the parsers (not at the same length!), though not as distinctive as in the blue line, since the running times are already a little higher at this point.

The incredibly low running times can be explained with similar reasoning, as for the low running times of bottom up on strings starting with a.
Top down is even faster, since it does not fill $tab$ completely.
In fact, when parsing strings not starting in a for the grammar of strings starting in a, it will only ever look at the most left children of the recursive tree, since everything returns falls
This results in a ver low amount of recursive calls.
In fact, the number of calls on the recursive function is $\Theta(n)$ (figure~\ref{fig:c_sa_td}).
As we argued in section~\ref{sec:top_down}, the upper bound for this number is $O(n^3)$.
The numbers for the counter of bottom up (repetitions of the inner most loop, i.e. over splitting points k) for parsing strings for the grammar of strings starting in a, are somewhere in the middle of $n^2$ and $n^2$, still yielding relatively fast running times.

\begin{figure}[h!]
    \centering
    \includegraphics[width=0.6\textwidth]{Images/c_sa_td.jpg}
    \caption{Running time (ms) of the top-down algorithm when parsing two set of strings of sizes 100-5000, in steps of hundred, for the Language of words starting in a.}
    \label{fig:c_sa_td}
\end{figure}

\todo{Add subsection for evaluations of step 3}




\newpage
\section{Conclusion}

Evaluating the non-specialized bottom-up and top-down algorithms showed multiple things.
Firstly, bottom-up has steadier running times than top-down.
This means, that even though the times may vary for different strings of the same length for the same grammar, they do not vary as much as for the top-down parser.
Further, it's worst case times are better, than those of top-down.
However, if the top-down parser is run on a favorable combination of a good ordering of the rules and form of input string, its running time may be very low.
Thus, its best-case running time is a lot lower than that of bottom-up.

The analysis further showed, that the running times could often be explained, by analyzing how the memoization table is filled and how many accesses to the table are performed in order to solve the problem.
Thus, if the input string is known, the faster algorithm can be detected with reasoning.

In general, it is safe to say, that both algorithms perform better if the grammar produces words with a distinctive property in the beginning, rather than in the end.
Parsing strings for the grammar of strings starting in a, different running times were yielded for different input strings of the same length.
Contrarily, for the grammar of strings ending in a, both the top-down and bottom-up parser yielded the same running times for any input string of same length, being slower than for any input string on the grammar for strings starting in a.

Analyzing the transformation of a linear grammars to an equivalent grammar in CNF can easily be done.
However, it does not improve the running time compared to parsing an equivalent grammar which is already in CNF.
Specializing the bottom-up algorithm on the other hand improves the running time a lot, as the algorithm is not in $O(n^3)$ anymore, but in $O(n^2)$.

It would be interesting to analyze, how the specialized algorithm performs on different grammars.
We could transform the grammars for strings starting and ending in a to equivalent linear grammars, and verify whether it still holds, that the running times are better if the distinctive property is in the beginning.


\pagebreak
\printbibliography

\end{document}
